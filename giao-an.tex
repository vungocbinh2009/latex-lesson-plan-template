\documentclass[12pt,a4paper]{article}
\usepackage[utf8]{inputenc}
\usepackage[T1]{fontenc}
\usepackage[utf8]{vietnam}
\usepackage{amsmath}
\usepackage{amsfonts}
\usepackage{amssymb}
\usepackage{graphicx}
\usepackage{booktabs}
\usepackage{multirow}
\usepackage{array}
\usepackage{ltablex}
\keepXColumns
\usepackage[left=3cm, right=3cm, top=2cm, bottom=3cm]{geometry}
\usepackage{titlesec}
\titlelabel{\thetitle. }
\usepackage{makecell}
\renewcommand\theadfont{\bfseries}
% Các môi trường dùng để gõ bài tập
\newtheorem{baitap}{Bài tập}
\newtheorem{loigiai}{Lời giải bài tập}

% Tạo dòng kẻ chấm.
\usepackage{setspace}
\def\dotfill#1{\cleaders\hbox to #1{.}\hfill}
\newcommand\dotline[2][.5em]{\leavevmode\hbox to #2{\dotfill{#1}\hfil}}
\author{Vũ Ngọc Binh}
\begin{document}
	\begin{center}
		\textbf{\LARGE Giáo án mẫu}
	\end{center}
	\begin{tabularx}{\linewidth}{ m{0.5\linewidth} X }
		Ngày soạn: a/b/c
		& 
		Tiết: d
		\\
		Ngày dạy: a/b/c
		& 
		Số tiết: e
		\\
		Người soạn: Vũ Ngọc Bình
		& 
		Lớp: f
		\\
	\end{tabularx}
	\section{Mục tiêu bài học}
	Sau khi học bài học này, học sinh sẽ:
	\subsection{Về kiến thức}
	\begin{itemize}
		\item Đây là một mục tiêu về kiến thức
	\end{itemize}
	\subsection{Về kĩ năng}
	\begin{itemize}
		\item Đây là một mục tiêu về kĩ năng
	\end{itemize}
	\subsection{Về thái độ}
	\begin{itemize}
		\item Đây là một mục tiêu về kĩ năng
	\end{itemize}
	\section{Chuẩn bị của giáo viên và học sinh}
	\subsection{Chuẩn bị của giáo viên}
	Bạn đã chuẩn bị những gì trước khi đến lớp ?
	\subsection{Chuẩn bị của học sinh}
	Học sinh cần chuẩn bị những gì trước khi nghe bạn giảng bài ?
	\section{Các phương pháp dạy học}
	Liệt kê các phương pháp dạy học tại đây.
	\section{Tiến trình bài học}
	\subsection{Hoạt động 1: Hoạt động vui chơi giải trí (45 phút)}
	\begin{tabularx}{\linewidth}{ | p{0.28\linewidth} | p{0.28\linewidth} | X | }
		\hline
		\endfoot
		\hline
		\thead{Hoạt động của GV}
		& 
		\thead{Hoạt động của HS}
		& 
		\thead{Ghi bảng - trình chiếu}
		\\ \hline
		\endhead
		Giáo viên làm gì đầu tiên
		&
		Học sinh làm gì đầu tiên
		&  
		Ghi nội dung lên bảng
		\\ \hline
		Giáo viên làm gì tiếp theo
		& 
		Học sinh làm gì tiếp theo
		&
		Ghi nội dung lên bảng
		\\ \hline
		Cuối cùng giáo viên làm gì
		&
		Cuối cùng học sinh làm gì
		&
		Ghi nội dung lên bảng
		\\ \hline
	\end{tabularx}

	\section{Phụ lục}
	\subsection{Phiếu bài tập}
	Ghi nội dung phiếu bài tập vào đây, sử dụng môi trường baitap
	\begin{baitap}
		Ghi bài tập vào đây
	\end{baitap}
	\subsection{Nội dung phiếu bài tập}
	Ghi nội dung lời giải các bài tập vào đây, sử dụng môi trường loigiai
	\begin{loigiai}
		\hfill
		
		Ghi lời giải tại đây
	\end{loigiai}

	\section{Rút kinh nghiệm}
	{\setstretch{2.0}
		\dotline{\linewidth}
		\dotline{\linewidth}
		\dotline{\linewidth}
		\dotline{\linewidth}
		\dotline{\linewidth}
		\par
	}
\end{document}
